The detection and measurement of protoplanets still embedded in the circumstellar disks they formed from is a vital step in constraining models of planet formation.
A promising method for making such measurements is through identifying kinematics features from perturbing planets in observations of molecular line emission in disks.
In this thesis we have built upon semi-analytic methods for modelling the interaction between the planet and the gas disk, and applied this methods to real observations to constrain potential planets.

In Chapters~\ref{ch:disks} and \ref{ch:planetdisk} were reviewed the relevant physics of both protoplanetary disk structure and planet-disk interactions.
We derived the linear disk response to a small perturbation using WKB methods, and found the the Lindblad resonance locations where a tidally-forcing body excited density waves.
Combining these findings we then used phase arguments to derive the shape of the coherent, one-armed wake that is formed from constructive interference of individual modes.
We then reviewed the work of \citet{goodman2001,rafikov2002a} in developing a semi-analytic framework to calculate the density perturbations along the planet wake, in both the linear and non-linear regimes.
Lastly, we applied this framework to determine the quantities that we must constrain observationally in order to measure planet masses.

In Chapter~\ref{ch:wake_models} we presented our Python package \textsc{wakeflow} for generating semi-analytic models of planet wakes in protoplanetary disks.
The package contains both accuracy and efficiency improvements over previous methods.
The accuracy improvements include a more accurate treatment of the initial conditions used in the non-linear wake evolution, improved matching between the linear and non-linear solutions, and a high-order method for extracting the velocity perturbations.

In Chapter~\ref{ch:HD169_IMLUP} we presented two applications of the models to the detection of protoplanets, in the disks of HD~169142 and IM~Lupi.
We found that the kinematic arc identified in HD~169142 is not well explained by the predominantly radial motions in the planet wake.
In IM~Lupi, we found that a spiral structure could be traced through the peak velocity map, providing evidence of an embedded planet.

Finally, in Chapter~\ref{ch:fitting} we presented preliminary work on the development of a fitting procedure to determine planet masses from kinematic observations.
To do this, we fit iso-velocity contours generated by semi-analytic models to the observed peak velocity map.
We found that this performed reasonably well for fitting just an unperturbed disk, but failed once a planet was added due to complications with how the distances between the curves in the model and observations were calculated.
We then presented a model-independent way of extracting the perturbations present in the kinematics, although it remains unclear how to leverage this to perform fitting of the planet mass.
However, this method does allow for quantitative identification of asymmetric features in the disk such as planet wakes, without needing to assume a background model.

\section{Future Work}