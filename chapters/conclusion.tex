The detection and measurement of protoplanets still embedded in the circumstellar disks they formed from is a vital step in constraining models of planet formation.
A promising method for making such measurements is through identifying kinematics features from perturbing planets in observations of molecular line emission in disks.
In this thesis we have built upon semi-analytic methods for modelling the interaction between the planet and the gas disk, and applied these methods to observations to constrain potential planets.

In Chapters~\ref{ch:disks} and \ref{ch:planetdisk} we reviewed the relevant physics of both protoplanetary disk structure and planet-disk interactions.
We derived the linear disk response to a small perturbation using WKB methods, and found the Lindblad resonance locations where a tidally-forcing body excited density waves.
Combining these findings we then used phase arguments to derive the shape of the coherent, one-armed wake that is formed from constructive interference of individual modes.
We then reviewed the work of \citet{goodman2001,rafikov2002a} in developing a semi-analytic framework to calculate the density perturbations along the planet wake, in both the linear and non-linear regimes.
Lastly, we applied this framework to determine the quantities that we must constrain observationally in order to measure planet masses.

In Chapter~\ref{ch:wake_models} we presented our Python package \textsc{wakeflow} for generating semi-analytic models of planet wakes in protoplanetary disks.
The package contains both accuracy and efficiency improvements over previous methods.
The accuracy improvements include a more accurate treatment of the initial conditions used in the non-linear wake evolution, improved matching between the linear and non-linear solutions, and a high-order method for extracting the velocity perturbations.

In Chapter~\ref{ch:HD169_IMLUP} we presented two applications of the models to the detection of protoplanets, in the disks of HD~169142 and IM~Lupi.
We found that the kinematic arc identified in HD~169142 is not well explained by the predominantly radial motions in the planet wake.
In IM~Lupi, we found that a spiral structure could be traced through the peak velocity map, providing evidence of an embedded planet.

Finally, in Chapter~\ref{ch:fitting} we presented preliminary work on the development of a fitting procedure to determine planet masses from kinematic observations.
To do this, we fit iso-velocity contours generated by semi-analytic models to the observed peak velocity map.
We found that this performed reasonably well for fitting just an unperturbed disk, but failed once a planet was added due to complications with how the distances between the curves in the model and observations were calculated.
We then presented a model-independent way of extracting the perturbations present in the kinematics, although it remains unclear how to leverage this to perform fitting of the planet mass.
However, this method does allow for quantitative identification of asymmetric features in the disk such as planet wakes, without needing to assume a background model.

\section{Future Work}

Further improvements to the semi-analytic models are likely needed before they can be treated seriously in the high-mass regime of more than a few thermal masses.
There are three main issues that need to be addressed in the case of large planet masses:
\begin{enumerate}
    \item The spatial discontinuity over the edge of the linear box, which becomes worse for increasingly more massive planets (see Section~\ref{sec:high_mass}). This effect is not physical, and so limits the application of the results from the models nearby the planets. The exact cause of the discontinuity should be investigated, so that it may be minimised if possible.
    \item The shape of the wake diverges from that predicted by purely linear theory \citep{ogilvie2002}. \citet{cimerman2021} introduced a correction term to account for this, and it should be possible to incorporate this into the solution.
    \item Large planets open a gap in the gas density profile centred on the planet orbital radius \citep[e.g.][]{ward1997}. This effect is currently not captured in the analytic models but is important for two reasons. Firstly, the resulting pressure gradient at the edge of the gaps results in a change to the azimuthal velocities in the disk \citep{teague2018}. Secondly, the $t$ coordinate that determines the rate of evolution along the wake depends on the surface density profile, and so a gap may effect the $t$ coordinate where the shock forms. The planet gap could be added using the analytic prescription from \citet{kanagawa2015}, and the effects of the pressure gradient included in the resultant velocities.
\end{enumerate}

Additional work also needs to be done for the planet mass fitting.
We found that the method attempted here, of fitting a model including both the background disk and planet perturbations projected to the emitting layer, did not produce a minimum in $\chi^2$ at the true planet parameters.
We therefore propose that further work should be done investigating model-independent methods of extracting the perturbations in the data, such as the preliminary work on mirror residuals we presented in Section~\ref{sec:mirror_residuals}.

Finally, the semi-analytic plus radiation transfer method presented in Section~\ref{sec:hd169} could be used to perform a study on the detectability of planet wakes in kinematic observations.
Since rotation curves are steeper towards the centre of a disk, velocity perturbations result in larger spatial deflections of the iso-velocity curves in the outer disk.
Furthermore, since the perturbations induced by the wake are predominantly radial \citep{rafikov2002a,calcino2022}, the azimuthal location of the planet should also affect detectability.
A detailed quantification of the orbital radii, azimuthal location of the planet, disk inclination, disk positive angle, and planet mass should be performed to determine the regions of parameter space which this method may be used to probe.